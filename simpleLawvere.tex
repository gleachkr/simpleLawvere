\documentclass{article}
\usepackage{amsmath, amsthm, amsfonts, tikz-cd}

\newtheorem{definition}{Definition}
\newtheorem{theorem}{Theorem}
\newtheorem{corrolary}{Corollary}
\newtheorem{conjecture}{Conjecture}
\DeclareMathOperator{\Id}{Id}
\DeclareMathOperator{\Lazy}{Lazy}

\begin{document}


\begin{definition}[Terminal]
  An object $\mathbf{1}$ is terminal if for every object $B$ there's a unique
  map $B\xrightarrow{!} \mathbf{1}$
\end{definition} 
A common intuition is that ``elements'' of objects correspond to arrows
$\mathbf{1}\rightarrow A$. When we have an arrow $\mathbf{1}\xrightarrow{x} A$,
we'll call it an element of $A$, and when we compose $f\circ x$, we'll write
$f(x)$. Because there's just one map $\mathbf{1}\rightarrow\mathbf{1}$,
$\mathbf{1}$ has only one element. We can denote this element by $\bullet$.

\begin{definition}[Product]
An product object for $A,B$ is an object $A\times B$ equipped with a pair of
``projection'' maps $A\times B\xrightarrow{\pi_A} A$, $A\times
B\xrightarrow{\pi_B} B$ such that for any pair of arrows $C\xrightarrow{f} A$
and $C\xrightarrow{g} B$, there's a unique map $C\xrightarrow{f\times g} A\times
B$ making the following diagram commute:
\begin{center}
\begin{tikzcd}
    && A \\
  C \ar[r,"f\times g"] \ar[urr,bend left=25, "f"] \ar[drr,bend right=25,"g" below]&A \times B\ar[ur,"\pi_A"]\ar[dr,"\pi_B~" below] \\
    && B
\end{tikzcd}
\end{center}
\end{definition}
One intuition is that a product is the set all the pairs $\langle a,b\rangle$ of
elements of $A$ and $B$, and the projections are defined by $\pi_A(\langle a,
b\rangle) = a$, $\pi_B(\langle a, b\rangle) = b$. If $a,b$ are elements of $A,B$
(i.e. arrows $\mathbf{1}\xrightarrow{a} A$, $\mathbf{1}\xrightarrow{b} B$) we
can respect this intuition by using $\langle a,b\rangle$ to refer to
$\mathbf{1}\xrightarrow{a\times b} A\times B$. We'll also write $f(a,b)$ instead
of $f(\langle a, b\rangle)$.

\begin{definition}[Exponential]
  An exponential object for $A,B$ is an object $A^B$ equipped with an
  ``application'' map $A^B\times B \xrightarrow{\mathsf{app}} A$ such that for
  any arrow $C\times B \xrightarrow{f} A$, there's a unique map $C\xrightarrow{\tilde{f}}A^B$ such
  that the following diagram commutes:
  \begin{center}
  \begin{tikzcd}
    A^B\times B \ar[r,"\mathsf{app}"]& A\\
    C\times B\ar[ur,"f" below]\ar[u, "\tilde{f}\times\Id_B"]&\\
  \end{tikzcd}
  \end{center}
\end{definition}
The intuition here is that $\tilde{f}$ is ``enclosing'' an element $c$ of $C$ in
a function $\lambda b.f(c,b)$, and that $\mathsf{app}$ applies functions
$\lambda b. \rho(b)$ in such a way that $\mathsf{app}(\lambda b.\rho(b), x)
= \rho(x)$. If $c$ is an element $\mathbf{1}\xrightarrow{c} C$ of $C$, we can
respect this intuition by using $\lambda b.f(c,b)$ to denote $\tilde{f}(c)$. If
$C$ is $\mathbf{1}$, we can abbreviate $\lambda b.f(\bullet,b)$ as just
$\lambda b. f(b)$
\begin{definition} A Cartesian closed category is one in which every pair of
  objects has a product and an exponential, and in which there's a terminal
  object.
\end{definition}

\begin{definition}\
  \begin{enumerate}
    \item If $i\circ f = i\circ g$, then write $f\simeq_i g$
    \item An object $A$ is homoiconic for $E\xrightarrow{i}Y$ if there exists an
      ``interpreter'' $A\xrightarrow{g}E^A$ such that for every $A\xrightarrow{f} E$,
      there is a ``code'' $\hat{f}$ for $f$, which is an element of $A$ such that
      $i\circ g(\hat{f}) = i\circ \lambda a . f(a)$
    \item An arrow $E\xrightarrow{i}Y$ has the fixed-point property if for every
      $E\xrightarrow{t} E$, there's an element $a$ of $E$ such that $i\circ t(a) = i(a)$.
    \item if $A$ is homoiconic for $E\xrightarrow{\Id_E}E$ we can just say that
      it is homoiconic for $E$
    \item if $E\xrightarrow{\Id_Y} E$ has the fixed-point property, then we can
      just say that $E$ has the fixed-point property.
  \end{enumerate}
\end{definition}

\begin{theorem}[Generalized Lawvere] In an Cartesian closed category, if $A$ is homoiconic for $E\xrightarrow{i}Y$ then $E\xrightarrow{i}Y$ has the fixed-point property.
\end{theorem}
\begin{proof}
  Suppose $A$ is homoiconic for $E\xrightarrow{i}Y$. Let $\bar{g}$ denote $\mathsf{app} \circ
  (g\times \Id_A)$. Then for any $A\xrightarrow{f}E$, there's an element
  $\hat{f}$ of $A$ such that for every element $a$ of $A$, 
  \begin{align*}
    i\circ \bar{g}(\hat{f}, a) &= i\circ \mathsf{app}(g(\hat{f}),a)  \\
                        &= i\circ \mathsf{app}(\lambda x. f(x),a)\\
                        &= i\circ f(a)
  \end{align*}
  now consider any $E\xrightarrow{t}E$. Let $k$ be the composition:
  \[\underbrace{A\xrightarrow{\Id_A\times Id_A} A\times A\xrightarrow{\bar{g}}E\xrightarrow{t}E}_k\]
  then for any element $a$ of $A$,
  \[\bar{g}(\hat{k}, a) \simeq_i k(a) = t(\bar{g}(a,a))\]
  But then if we set $a$ to also be equal to $\hat{k}$ we see that $\bar{g}(\hat{k},\hat{k})\simeq_i t(\bar{g}(\hat{k},\hat{k}))$. Since $t$ was arbitrary, $E\xrightarrow{t}Y$ has the fixed point property.
\end{proof}

Note that in the category of sets, an interpreter $A\xrightarrow{g}Y^A$ would be
a surjection. So

\begin{corrolary}[Cantor/Lawvere] In the category of sets, $\mathbf{2}$ doesn't have the
  fixed-point property (consider the map that swaps elements). So, there can
  never be a surjection $A\rightarrow \mathbf{2}^A$, since that would amount to
  $A$ being homoiconic for $\mathbf{2}$
\end{corrolary}

\begin{conjecture}[Kleene] If we have a category where
  $\mathbb{N}\rightarrow\mathbb{N}$ is the set of total recursive functions,
  then $\mathbb{N}$ is homoiconic for $\mathbb{N}\xrightarrow{\phi}\mathcal{F}$,
  where $\mathcal{F}$ is the set of partial recursive functions and $\phi$ is 
  given $e\mapsto\phi_e$.

  Let $\Lazy(x,y)$ be the total function that maps each $x,y$ to the program
  \begin{verbatim}
  let e = phi(x,y)
  return apply(e,$1)
  \end{verbatim}
  then $g:x\mapsto \lambda y.\Lazy(x,y)$ is an interpreter. If
  $\mathbb{N}\xrightarrow{k}\mathbb{N}$ is an
  arbitrary total recursive function, then we do have a code $\hat{k}$ such that
  $\lambda y.\phi_{\Lazy(\hat{k},y)} = \lambda y.\phi_{k(y)}$; so $\mathbb{N}$ is
  homoiconic for $\phi$. So for every total recursive $k$, there's an element of
  $\mathbb{N}$ such that $\phi_{k(e)} = \phi_e $
\end{conjecture}

\end{document}
